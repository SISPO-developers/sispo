\section{Conclusion} \label{sec:conclusion}
\subsection{Summary}
It was possible to develop a first draft of the \gls{sispo} software package. \Gls{sispo} can simulate flyby scenarios around \gls{sssb}. Images are rendered and photometrically calibrated along the trajectory providing synthetic images of a \gls{sssb} surface. Using Ray-tracing techniques through the physics-based rendering engine Cycles, creates photo-realistic images. \Gls{sispo} can deal with a broad range of distances to the nucleus using procedural terrain generation which creates surface details as needed during the rendering process. Comparison with images from the asteroid Bennu and the comets \gls{67p} and \gls{81p} show that the current shader implementation creates imagery resembling mostly an asteroid rather than a comet due to a missing gas and dust model. The rendered images are then compressed with either a lossless or lossy compression technique, to reduce the data size at the cost of introducing artefacts. In the final step, the compressed images are used to reconstruct a \gls{3d} model leveraging \gls{sfm} techniques by using \gls{omvg} and \gls{omvs}. Three reconstruction algorithms were used during the simulation. While it seems that simple scenes are better reconstructed with the IncrementalSfM algorithm, the IncrementalSfM2 algorithm reconstructed the most points in most scenarios. The GlobalSfM algorithm only once exceeded the result of IncrementalSfM and IncrementalSfM2 hence the incremental \gls{sfm} approach is more successful in our application. It was found that compression introduces artefacts that can increase the number of reconstructed points from the \gls{sfm} algorithms if the artefacts are not too severe. For high compression ratios, the difference between lossy and lossless images does not resemble random noise anymore but surface features become visible, i.e. surface features are being removed by compression explaining a lower quality of the reconstructed models. Additionally, it seems that the compression slightly increases the brightness of images which might also improve the quality of reconstructed models. Considering only the number of faces of the reconstructed models as a quality measure, compression is beneficial for improving model quality.
Problems during rendering include fireflies, a few extremely bright pixels, and stripes were encountered.
Problems with reconstruction were found to be inverted \gls{3d} models and surfaces not being closed.
In the future, \gls{sispo} can aid in designing missions to \glspl{sssb}. The \textit{HERA} and the \textit{Comet Interceptor} missions target a \gls{sssb} with a mother spacecraft carrying a number of small satellites to these bodies. Therefore, investigating the effect of compression is a first step in maximising the useful scientific image data that can be transmitted from these small spacecraft. Also mission concepts like \textit{\gls{mantis}}, \textit{CASTAway} or \textit{\gls{mat}} could use \gls{sispo} for refining their design and maximising their data throughput. Additionally, \gls{sispo} can be used for developing optical navigation algorithms by creating a large number of images or improving camera pose estimates from the \gls{sfm} pipeline.

It was possible to show the influence of compression on images. However, quantifying image quality solely on the number of reconstructed points

\subsection{Future Developments}
While providing a foundation for developing \gls{sispo}, several issues could not be addressed in the course of this work.
First, there is currently no realistic model of spacecraft attitude motion and control implemented. The camera of the simulation environment is perfectly oriented towards the centre of the \gls{sssb}'s nucleus during the entire simulation. Realistic rotation should cover at least two effects, motion blur due to instantaneous rotation velocities of spacecraft and off-centre pointing due to control inaccuracies. Furthermore, it is necessary to include  image distortions such as astigmatism, bokeh, coma, field curvature, glare.

Currently, \gls{sispo} assumes that an instrument always has \gls{rgb} channels. Since many \gls{ccd}s used in deep space are only \gls{bw} cameras, a possibility to choose either \gls{rgb} or \gls{bw} should be implemented.

Moreover, it is necessary to include a gas and dust environment around the nucleus. From a technical perspective, a proper simulation of the data transmission should be included. For example, a realistic simulation for packet loss. The ultimate goal is to develop a prioritisation algorithm for the images which should prioritise data transmission on packet level.
Moreover, multi-instrument capability was intended as well as including multiple \gls{sssb}s in the future to allow more complex simulations including e.g. a binary system.

Furthermore, the shader used to create the \gls{sssb} models should be developed further. The interface for it should be included into \gls{sispo} and restricted to values that create reasonable shaped outputs. Additionally, the shader should also represent comet surfaces better. Furthermore, since most of the execution time is spent rendering, the shader should be developed to be less computationally heavy.

An attempt to include a HAPKE model via the synthspace package \url{https://github.com/oknuutti/synthspace} was unsuccessful. However, it would be interesting to compare the results of Blender and the HAPKE model. A HAPKE model is substantially faster while being less accurate though possibly still sufficient for some cases.

To improve computational performance and make image compression and possible data transmission more realistic, image cropping should be added. A substantial part of an image far from a nucleus is black which could be cropped away to reduce the amount of data for the computation and the system itself.