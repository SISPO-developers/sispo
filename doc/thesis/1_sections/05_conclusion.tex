\section{Conclusion} \label{sec:conclusion}
%\subsection{Summary}
The original code base which included rendering and reconstruction capabilities was developed further into a first version of the \gls{sispo} software package. Furthermore, the software package was extended to include image compression and decompression capabilities. \Gls{sispo} was used to simulate several fly-by scenarios with \gls{sssb} sizes of \SI{1}{\kilo\meter} and \SI{10}{\kilo\meter} and fly-by distances of \SI{50}{\kilo\meter}, \SI{100}{\kilo\meter}, \SI{200}{\kilo\meter} and \SI{400}{\kilo\meter}.

Comparison with images from the asteroid Bennu and the comets \gls{67p} and \gls{81p} show that the current shader implementation creates imagery resembling mostly an asteroid rather than a comet due to a missing gas and dust model and other missing types of surface features.

Rendered images are compressed with either \gls{png} or \gls{jp2} which represent a lossless or lossy compression technique respectively. Lossy compression reduces the amount of data more than lossless compression at the cost of introducing artefacts. While lossy compression with low compression ratios introduces artefacts which resemble random noise, artefacts introduced by high compression ratios show correlation to surface features. Visually inspection confirms that low compression ratios do not degrade specific parts of an image in contrast to high compression ratios which clearly blur contours of surface features.

The compressed images are used to reconstruct a \gls{3d} model leveraging \gls{sfm} algorithms provided by \gls{omvg} and \gls{omvs}. Three reconstruction algorithms were used during the simulation. While simple scenes were better reconstructed with the IncrementalSfM algorithm, the IncrementalSfM2 algorithm reconstructed the most points in most scenarios. The GlobalSfM algorithm reconstructs more points than IncrementalSfM and IncrementalSfM2 in only one case hence the incremental \gls{sfm} approach is more successful in our application. A \gls{3d} model could successfully be reconstructed in all tested cases except for a \SI{400}{\kilo\meter} fly-by of a \SI{1}{\kilo\meter} \gls{sssb} with highest compression ratio. It was found that lossy compression introduces artefacts that can increase the number of reconstructed points from the \gls{sfm} algorithms if the artefacts are not too severe. High compression ratios alter surface features which leads to a lower quality of the reconstructed models. 

Several problems were found during rendering and reconstruction. One rendering problem are stripes, dark patches across the entire \gls{sssb} nucleus which were found on many images of a \SI{10}{\kilo\meter} \gls{sssb}. Another less severe problem that was encountered are  fireflies, a few extremely bright pixels, which were only found in one image of a \SI{50}{\kilo\meter} fly-by of a \SI{10}{\kilo\meter} nucleus. Both problems could not be resolved in this work.
Reconstruction problems were found to be inverted \gls{3d} models and surfaces not being closed. These could be tackled by removing optimisation of intrinsic camera parameters which can be justified with the assumption of known calibration values for imagers in space missions. Additionally, motion priors, i.e. initial guesses for extrinsic camera parameters, were provided to the \gls{sfm} pipeline to decrease the probability of inverted models. The use of motion priors is justified since a rough trajectory is always known for space missions.

\Gls{sispo} can aid in designing missions to \glspl{sssb}. The \textit{HERA} and \textit{\gls{ci}} missions target a \gls{sssb} with a mother spacecraft carrying a number of small satellites. Therefore, investigating the effect of compression is a first step in maximising the useful scientific image data that can be transmitted from these small spacecraft. Mission concepts like \textit{\gls{mantis}}, \textit{CASTAway} or \textit{\gls{mat}} could use \gls{sispo} to improve their design and maximising their data throughput. Additionally, \gls{sispo} can be used for developing optical navigation algorithms. Either indirectly by creating a large number of images or directly by utilising the camera pose estimates provided by the \gls{sfm} pipeline.

%\subsection{Future Developments}
While providing a foundation for \gls{sispo}, several issues could not be addressed in the course of this work. First, a realistic model of spacecraft attitude motion and control is missing. The camera of the simulation environment is perfectly oriented towards the centre of the nucleus during the entire simulation. Realistic rotation should cover at least two effects, motion blur due to instantaneous rotation velocities of spacecraft and off-centre pointing due to control inaccuracies. Furthermore, it is necessary to include imaging distortions such as astigmatism, bokeh, coma, field curvature, glare.

Moreover, it is necessary to include a gas and dust environment around the nucleus to extend the rendering capabilities from asteroids to comets. Including multiple \glspl{sssb} in the future would allow even more complex simulation scenarios including e.g. a binary system. Furthermore, a more recent and accurate star catalogue, like the GAIA catalogue, could be implemented to improve star map rendering.

Currently, \gls{sispo} assumes that an instrument always uses \gls{rgb} channels. Since many imagers used in deep space use monochrome \glspl{ccd}, an option to select either \gls{rgb} or monochrome output should be implemented. In addition, using an improved photometric system, such as the \gls{ubvri} \cite{Bessell1993PhotometricSystems}, that takes into account the sensitivity of \gls{ccd} in the red and infrared spectrum would improve realism of the rendered images.

A simulation of data transmission should be included. For example, a realistic simulation for packet loss using common radio transmission methods and protocols. Additionally, compression techniques which are commonly used in deep space missions should be implemented. Effects of non-optimal lighting conditions, i.e. over and under-exposure, on compression should be investigated. Image cropping should be added to increase the realism of image compression and data transmission. A substantial part of an image far from a nucleus is black which could be cropped away to reduce the amount of data. The ultimate goal, e.g. for the \textit{\gls{ci}} mission, would be to develop a prioritisation algorithm for the images which can prioritise data transmission on packet level.

Furthermore, the shader that implements the procedural terrain generation should be developed further. The interface for it should be included into \gls{sispo} and restricted to values that create reasonable shaped outputs. Additionally, the shader should also represent a comet's surfaces better. Furthermore, since most of the execution time is spent rendering, the shader should be developed to be less computationally heavy.

An attempt to include a Hapke model via the synthspace package\footnote{\url{https://github.com/oknuutti/synthspace}} was unsuccessful. However, it would be interesting to compare the results of \gls{sispo} and the Hapke model. A Hapke model is substantially faster while being less accurate. One possible use case would be creating real-time imagery which is currently not possible with \gls{sispo}.