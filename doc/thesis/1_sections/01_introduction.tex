\section{Introduction} \label{sec:introduction}
%Deep Space Missions
%-science (planetary defence)
%    -navigation autonomy
%    -maximising science output
%        -high risk environments
%        -small data budgets
%            -compression
%    -CI
%        -unknown object, small real image database
%-economy
%    -navigation autonomy
%    -knowledge about composition, size and shape -> science
%==> autonomy, max sci, unknown object -> synthetic image generation, compression and reconstruction -> sispo covers all three topics

% SSSBs Motivation:
% - Scientifically interesting
%   - remnants of solar system formation
%   - asteroids: 
%     - remnants of formation of inner planets
%     - rocky
%     - variety
%   - comets:
%     - bombardment
%     - building blocks of life
%     - icy
% - potential threat
% - ISRU
% Remotely many classified, better to improve statistics of up close study 

%SSSBs visited: Ryugu, Bennu, Ceres, C-G, Vesta, Tempel 1, Hartley, Lutetia, Itokava, Steins, Wild 2, Annefrank, Borrelly, Eros, Braille, Mathilde, Ida, Gaspra, Halley, Giacobini-Zinner

% Small sat asteroid: NEA Scout

Studying \glspl{sssb} provides a unique opportunity to understand the evolution of the Solar System, since asteroids, comets and other \glspl{sssb} are remnants of the early formation phase~\cite{walsh2018rubble, a2017comets}. They are being studied to understand the delivery of water and carbon-based molecules to Earth and other planets. While asteroids can be studied to understand the chemical composition of the inner Solar System, long- and short-period comets originate from the \"Opik-Oort cloud and can be studied to understand the chemical composition of the outer and early Solar System.

Most information is obtained using remote measurements with ground based telescopes or space telescopes~\cite{bowles2018castaway}. This method can provide statistics for thousands of objects with the same instrument. It allows observing their orbits, albedos, colors, spectra and sizes. However, space missions can be used to create detailed maps of albedos, colors, spectra and surface features, \gls{3d} models, precise mass and density estimates, physical properties and chemical composition, both surface material composition and sub-surface composition. Several missions to \glspl{sssb} are planned. For example, the \textit{\gls{ci}} mission by \gls{esa} will target an unknown dynamically-new comet~\cite{snodgrass2019europeanCI}, the \textit{HERA} mission by \gls{esa} and the \textit{\gls{dart}} mission by \gls{nasa} to the double asteroid (65803) Didymos (1996 GT)~\cite{hera, talbert_2017DART}.

More than \SI{3600}{} comets and \SI{930000}{} asteroids are known~\cite{nasaSBD_count, mpc2020}. Most knowledge about their size, shape and colour is inferred from remote measurements via ground based observations and space telescopes. The largest group of \glspl{sssb}, asteroids, are being categorised into \SI{24}{} categories because of their variety in size, spectral type, surface activity and other characteristics~\cite{demeo2009extension}. Approximately \SI{20}{} of such objects have been visited by spacecraft~\cite{nasaSBD_missions}. These missions were monolithic, single spacecraft missions which targeted one or a few \glspl{sssb} each. Often these missions were designed as orbiters around one or two \glspl{sssb}. For example the \textit{Hayabusa 2} mission targeted asteroid (162173) Ryugu (1999 JU3) or the \textit{OSIRIS-REx} mission which orbited (101955) Bennu~\cite{Watanabe2017Hayabusa2Overview, lauretta2017osiris}. With costs around \$1~billion for each mission, it seems infeasible to visit many targets and study the large variety there is.

Another approach is to perform flybys of several objects which increases the number of targets at the cost of being limited in observation time. For example, the \textit{Lucy} mission will fly by several of Jupiter's Trojans~\cite{stanbridge2017lucy} or the \textit{CASTAway}, \textit{\gls{mantis}} or the \textit{\gls{mat}} missions concepts which aim at touring main belt asteroids~\cite{bowles2018castaway, rivkin2016mainmantis}. To increase the number of targets even more, small spacecraft can be used. Rapidly developing technology provides more opportunities to study such objects since interplanetary missions become feasible at a reasonable cost using small spacecraft~\cite{Poghosyan2017CubeSatMissions, andrews2019asteroid, snodgrass2019europeanCI}. For example, the \textit{\gls{mat}} mission proposes to fly by hundreds of main belt asteroids using a fleet of small spacecraft~\cite{Slavinskis2018NanospacecraftSails}. The \textit{HERA} mission will deploy its daughter-spacecraft \textit{APEX} which will autonomously navigate at the Didymos system~\cite{Kohout2018FeasibilityStudy}. For the \textit{\gls{ci}} mission, two 'sub-spacecrafts' will independently fly-by a target and perform measurements\cite{snodgrass2019europeanCI}.

Using small and low-cost spacecraft, it becomes possible not only to have more spacecraft to study \glspl{sssb} but also to fly closer to their nuclei, which is scientifically more interesting but associated with a higher risk of losing a spacecraft. This puts two major constraints onto their data budget. Due to size and power limitations, miniature radios provide lower data rates. Combined with the possibility of losing the spacecraft means transmitting the most important data first and complementing it later, if possible. Compression is used to reduce the amount of data that needs to be transmitted. The two main categories of compression are lossless compression and lossy compression. With lossless compression, no information is lost during the compression process. In contrast, lossy compression accepts a certain amount of loss of information to achieve higher compression ratios. Losing information in images results in artefacts, however depending on the method, these can be made nearly imperceptible. Therefore it is necessary to investigate the effect of compression, especially lossy compression, on image data. In addition, an algorithm is necessary that can prioritise which data needs to be transmitted first and which can be transmitted later.

Designing such deep space missions, maximising their science output and targeting unknown objects require a versatile simulation environment to develop, test and validate systems and algorithms. The SurRender software offers such features. However, it is proprietary software developed by Airbus which cannot be accessed by external parties \cite{Brochard2018ScientificSoftware}. Another software suite available for \gls{esa} projects or commercially is \gls{pangu}. However, \gls{pangu} uses OpenGL 3.0 which is fundamentally limited in its accuracy \cite{Martin2019PlanetaryPANGU}. Other software suite such as the \gls{asp} only provide stereogrammetry capabilities for geodesy purposes and thus can only be used for missions to bodies with existing image archives \cite{Beyer2018TheData}.

Although data sets from the \textit{OSIRIS} instrument aboard \textit{Rosetta}~\cite{osirisArchive} or from the \textit{OSIRIS-REx}, \textit{Dawn} and \textit{NEAR} missions~\cite{palmer2014small} are publicly available, the total number of images and their variation remains small. Since there is only a limited number of real images it is necessary to synthetically create images of \glspl{sssb}. Synthetic image creation of \glspl{sssb} is currently based on the parametric empirical models for directional reflectance properties of airless regolith surfaces developed in a paper series by Hapke~\cite{Hapke1981BidirectionalTheory, hapke1981bidirectional2, Hapke1984BidirectionalRoughness, Hapke1986BidirectionalEffect, Hapke2002BidirectionalScattering, Hapke2008BidirectionalPorosity, Hapke2012Bidirectional7}. However, the Hapke model is being challenged because it is an empirical with shortcomings, for example with shadows especially at slopes~\cite{shkuratov2012critical}. Shadows are crucial to represent surface features of \glspl{sssb}. Ray-tracing techniques can be used to overcome such problems~\cite{shkuratov2012critical, lafortune1996mathematical}.

%Images produced by a mission to a \gls{sssb} can be used to reconstruct the \gls{3d} shape and surface through computer vision techniques. In addition, the number of reconstructed points can be used to quantify the quality of the images.

The \gls{sispo} is being developed to cover the three topics mentioned above. First, physics-based image rendering using Ray-tracing is implemented. Second, compression and decompression using different algorithms is performed to investigate the quality loss on the rendered images. Third, a \gls{sfm} processing pipeline is established to reconstruct a \gls{3d} model of the \gls{sssb}.

The aim of this project is to develop a first version of \gls{sispo} that covers a basic functionality of rendering, compression and reconstruction. Based on an initial \gls{3d} model, \gls{sispo} shall be able to render images of a realistic flyby scenario, compress and decompress these images to introduce compression artefacts and reconstruct a \gls{3d} model. This pipeline shall be used to investigate the effects of compression and the possible amount of data reduction. To see the effects of compression, a lossy compression technique with different compression ratios shall be compared to lossless compression. This will provide a first insight to which degree images can be compressed and how much the necessary data to be transmitted can be reduced.

The content of this thesis is structured into four main sections. \newline
\textbf{Section~\ref{sec:theory}~\nameref{sec:theory}} provides background information about \glspl{sssb}. Furthermore, important concepts of image rendering and processing are explained. Moreover, an introduction to a the \gls{sfm} technique is provided. The section concludes with background information about image compression and processing. \newline
\textbf{Section~\ref{sec:sispo}~\nameref{sec:sispo}} explains the implementation, design choices and input parameters of the \gls{sispo} software package. \newline
\textbf{Section~\ref{sec:results}~\nameref{sec:results}} presents rendered images, reconstructed \gls{3d} models and compression effects on these two. Moreover, an analysis of the results is presented along with a discussion. \newline
\textbf{Section~\ref{sec:conclusion}~\nameref{sec:conclusion}} summarises the content of the thesis and provides an outlook on possible future developments of \gls{sispo}.

% \begin{enumerate}
%     \item \textbf{Section~\ref{sec:theory} Theory} provides an introduction into the background of \glspl{sssb}, image rendering, compression techniques and \gls{sfm}.
%     \item \textbf{Section~\ref{sec:sispo} \gls{sispo}} elaborates on the implementation of the \gls{sispo} software package and design choices that were made.
%     \item \textbf{Section~\ref{sec:results} Results} shows results created by \gls{sispo} and their analysis.
%     \item \textbf{Section~\ref{sec:conclusion} Conclusion} concludes the document by summarising the content and providing an outlook on possible future developments with \gls{sispo}.
% \end{enumerate}

%All these endeavours lead to an increased number of spacecraft that operate in deep space. Current operations and navigation of spacecraft in deep space relies on \gls{dsn}. Radio signals are used to determine a spacecraft's state vector through ranging and Doppler measurements~\cite{ramamurthy2015delta}. Relying on such an infrastructure limits the number of spacecraft that can be operating in deep space. Additionally, this poses not only a risk to a mission due to reliance on a working ground segment, but is also associated with high costs. High costs are especially relevant to small spacecraft mission, since using a \gls{dsn} would increase the overall mission cost substantially~\cite{steffes2017deep}. Furthermore, when a spacecraft is obstructed by the Sun, it is not possible to communicate with it and therefore not possible to use such techniques~\cite{kominato2006optical}. Another problem with \gls{dsn}-based navigation become apparent with \textit{Comet Interceptor}, where high relative speeds do not allow for long delays associated with ground contacts in deep space. Similarly, with rendez-vous and landing missions to \gls{sssb}s, autonomous orbit determination is essential because mission events are happening too fast for relying on ground interaction~\cite{shuang2013imageprocessing}. An alternative to using a \gls{dsn} are \gls{dsan} technologies. \gls{dsan} comprises trajectory and attitude determination only from on-board data and with on-board available computational power. In a next step, this information would be fed into an \gls{aocs} to correct pointing and trajectory errors. For example, optical navigation can be conducted on-board the spacecraft thus eliminating the requirement of ground contact. Optical navigation was already used in the \textit{Deep Space 1} mission cruise phase (\cite{Riedel2000AutonomousReport},~\cite{bhaskaran2012autonomous}). However, autonomous relative optical navigation in the proximity of a Solar System body will be essential for future deep space exploration and space mining economy (\cite{steffes2017deep},~\cite{martin2006jpl},~\cite{probst2016mission}). Hybrid optical navigation was used in the \textit{Hayabusa} mission in the proximity of the asteroid Itokawa. However, this method still relied on radiometric data from the ground segment~\cite{kominato2006optical}. Therefore, fully autonomous navigation systems still need to be developed.

%In this thesis, the first implementation of \gls{sispo} is developed. The aim was not to complete development of the software package but to provide a first draft. The implementation covers the entire processing pipeline from rendering and compression to reconstruction of a 3D model. These sub components are combined to form \gls{sispo}. To create realistic observation geometries, trajectories are simulated with proper orbital dynamics in the solar system.

%The software package should provide a possibility to simulate different trajectories. Rendering should provide as realistic output as possible. Furthermore, it is necessary to include rendering of various \gls{sssb}s, including different size, shape and surface features. To assess the quality of rendered images and to investigate the impact of compression, it is posited that the quality recovered by an \gls{sfm} pipeline provides a quality measure in the form of the number of reconstructed points, number of vertices and number of faces. \gls{sispo} is being developed to provide these functionalities in a single software package.

%Advancing the field requires improving the statistics on the different types of \gls{sssb}s~\cite{Pajusalu2019CharacterizationMapping}. Several new mission ideas are aiming for this goal. However, technological challenges have to be overcome first. Additionally, new mission concepts like the \gls{ci} mission require a new approach to model \gls{sssb}s. \gls{ci} is a mission that will investigate either a dynamically new comet from the Öpik-Oort cloud or an interstellar object. The mission will consist of a main spacecraft and two sub-spacecrafts that will be launched to the \gls{l2} and wait until a suitable \gls{sssb} will pass by. Consequently, the target body will not be known before launch hence systems have to be developed to deal with a large variety of bodies. However, only few \gls{sssb}s have been visited by spacecraft and thus only very limited real images are available. A method to deal with such sparse available data is to synthetically create images of \gls{sssb}s.


%This thesis focuses on image compression and decompression. These effects will be quantified using the reconstructed 3D models against the "ideal" model. A reference instrument definition similar to a suggested instrument for the 'Comet Interceptor' mission. Different compression and decompression algorithms will be evaluated using different \gls{sssb}s. Furthermore, different spacecraft trajectories will be examined to investigate limitations of the simulation environment.
