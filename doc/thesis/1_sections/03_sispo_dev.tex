\section{\Acrlong{sispo}}

\gls{sispo} is a software package developed in python. It is separated in different sub-modules. The two main sub-modules are the 'simulation' and the 'reconstruction' module.

The software package is hosted on GitHub using a git version control system. Furthermore, the GitHub project management tools are used, including automated KanBan based projects and issue tracking.

The most important python dependencies of \gls{sispo} are:
\begin{itemize}
    \item astropy: Astronomy package developed by \cite{robitaille2013astropy} and \cite{price2018astropy}
    \item blender: 3D creation suite
    \item numpy: Scientific computing for python
    \item opencv: Computer vision library used for image processing
    \item OpenEXR: \gls{hdr} image reading and writing
    \item orekit: Space dynamics library
\end{itemize}

\subsection{simulation module}
The simulation module creates photo-realistic images by simulating a realistic trajectory of a \gls{sssb} and a spacecraft using orekit. This trajectory and attitude data is then used to render four images per frame, one containing only the \gls{sssb}, one where the view is kept at a constant distance from the \gls{sssb}, one calibration reference image and one that renders a realistic star background using the \gls{ucac4}. These images are composed to one image using photometry to calibrate the absolute light intensity of the different images in terms of realistic photon fluxes using the Johnson magnitude system \cite{bessel1979ubvri}.

All images during the rendering and calibration process use \gls{hdr} images to minimise the loss of information in intermediate steps.

\subsection{compression module}
The compression module provides compression and decompression algorithms. These can be tested against different mission scenarios and image series to investigate the impact of compression and decompression on the science quality.

The following set of compression algorithms 

\subsection{reconstruction module}
The reconstruction module can be used to generate a 3D model of an object using a series of images. It provides a full Multi-View Stereo reconstruction data process pipeline. To achieve this, two libraries are used and combined and called using python. The first library is \gls{omvg} by \cite{openMVG}. The second library is \gls{omvs}. 
The common steps for the complete pipeline is comprised of the following steps:
\begin{enumerate}
    \item Read in images [ImageListing in \gls{omvg}]
    \item Compute visual features [ComputeFeatures in \gls{omvg}]
    \item Match computed features between different images [MatchFeatures in \gls{omvg}]
    \item Generate point cloud from matched features [IncrementalSfM in \gls{omvg}]
    \item Export to \gls{omvs} format [openMVG2openMVS in \gls{omvg}]
    \item Increase number of points in point cloud [DensifyPointCloud in \gls{omvs}]
    \item Create a mesh from the point cloud [ReconstructMesh in \gls{omvs}]
    \item Refine the generated mesh [RefineMesh in \gls{omvs}]
    \item Apply texture to mesh to create final 3D model [TextureMesh in \gls{omvs}]
\end{enumerate}


\subsection{Future Developments}
There are several issues left open within the \gls{sispo} software package. First, there is currently no realistic model of spacecraft attitude motion and control implemented. The camera of the simulation environment is perfectly oriented towards the centre of the \gls{sssb}'s nucleus during the entire simulation. Realistic rotation should cover at least two effects, motion blur due to instantaneous rotation velocities of spacecraft and off-centre pointing due to control inaccuracies. Furthermore, it is necessary to include  image distortions such as astigmatism, bokeh, coma, field curvature, glare. Moreover, it is necessary to include a gas and dust environment around the nucleus. From a technical perspective, a proper simulation of the data transmission should be included. For example, a realistic simulation for packet loss. The ultimate goal is to develop a prioritisation algorithm for the images which should prioritise data transmission on packet level.