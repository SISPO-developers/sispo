\section{Introduction} \label{sec:introduction}
%Deep Space Missions
%-science (planetary defence)
%    -navigation autonomy
%    -maximising science output
%        -high risk environments
%        -small data budgets
%            -compression
%    -CI
%        -unknown object, small real image database
%-economy
%    -navigation autonomy
%    -knowledge about composition, size and shape -> science
%==> autonomy, max sci, unknown object -> synthetic image generation, compression and reconstruction -> sispo covers all three topics

%SSSBs visited: Ryugu, Bennu, Ceres, C-G, Vesta, Tempel 1, Hartley, Lutetia, Itokava, Steins, Wild 2, Annefrank, Borrelly, Eros, Braille, Mathilde, Ida, Gaspra, Halley, Giacobini-Zinner

The study of \gls{sssb} provides a unique opportunity for science to understand the development of the Solar System since they are unaltered remnants of the formation phase. Rapidly developing technology provides more opportunities to study such objects since interplanetary missions become feasible at a reasonable cost using small spacecraft. 

A total of \SI{3607}{} comets and \SI{931905}{} asteroids are known \cite{nasaSBD_count}. Most knowledge about there size, shape and colour is inferred from ground based observations and space telescopes \cite{bowles2018castaway}. Another method is close-up observation through spacecraft. Approximately \SI{25}{} of such objects have been visited by spacecraft (\cite{wikipediaVisitedList}, \cite{nasaSBD_missions}). Several missions to \gls{sssb} are planned. For example, the \textit{HERA} mission by \gls{esa} \cite{hera}, the \textit{Comet Interceptor} mission by \gls{esa} \cite{snodgrass2019europeanCI} and the \textit{DART} mission by \gls{nasa} \cite{talbert_2017DART}. Closer studies through flyby or orbiting an \gls{sssb} provides more detailed information such as precise size and shape measurements as well as local variations in albedo for example.

Using small and low-cost spacecraft it becomes possible not only to have more spacecraft to study \gls{sssb}s but also to fly closer to their nuclei, which is scientifically more interesting but associated with a higher risk of losing a spacecraft. This puts two major constraints onto their data budget. Their radio links provide lower data rates and the possibility of losing the spacecraft means transmitting the most important data first and complementing it later, if possible. To achieve this, an algorithm is necessary that can prioritise which data needs to be transmitted first and which can be transmitted later.

In addition, especially asteroids are becoming interesting to companies. Numerous studies show that asteroid mining can become a viable business in the near future (\cite{andrews2015defining}, \cite{busch2004profitable}, \cite{weinzierl2018EconomicFrontier}, \cite{pittman2017deep}). Several companies, such as Planetary Resources \cite{lewicki2013planetary} or the Asteroid Mining Corporation Ltd. \cite{asteroidminingcorporation} were founded for creating an asteroid mining based business. For such an economy to thrive, it is essential to have sound knowledge about the asteroid population but also to develop as autonomous spacecraft as possible.

All these endeavours lead to an increased number of spacecraft that operate in deep space. Current operations and navigation of spacecraft in deep space relies on \gls{dsn}. Radio signals are used to determine a spacecraft's state vector through ranging and Doppler measurements \cite{ramamurthy2015delta}. Relying on such an infrastructure limits the number of spacecraft that can be operating in deep space. Additionally, this poses not only a risk to a mission due to reliance on a working ground segment, but is also associated with high costs. Furthermore, when a spacecraft is obstructed by the Sun, it is not possible to communicate with it and therefore not possible to use such techniques \cite{kominato2006optical}. An alternative to using a \gls{dsn} are \gls{dsan} technologies. For example, optical navigation can be conducted on-board the spacecraft thus eliminating the requirement of ground contact. Optical navigation was already used in the \textit{Deep Space 1} mission cruise phase \cite{Riedel2000AutonomousReport}. However, autonomous relative optical navigation in the proximity of a Solar System body will be essential for future deep space exploration (\cite{steffes2017deep}, \cite{martin2006jpl}). Hybrid optical navigation was used in the \textit{Hayabusa} mission in the proximity of the asteroid Itokawa. However, this method still relied on radiometric data from the ground segment \cite{kominato2006optical}.

For developing navigation algorithms and methods to maximise scientific output, it is necessary to have images of \gls{sssb}s in order to test and validate these algorithms and methods. Although data sets from the \textit{OSIRIS} instrument aboard \textit{Rosetta} \cite{osirisArchive} or from the \textit{OSIRIS-REx}, \textit{Dawn} and \textit{NEAR} missions \cite{palmer2014small} are publicly available, the total number of images and their variation remains small.

Increased spacecraft autonomy around \gls{sssb}, maximising the science output and targeting unknown objects require images from a variety of differing objects to develop systems and algorithms. Since there is only a limited number of real images it is necessary to synthetically create images of \gls{sssb}s. To develop an algorithm that can be used to prioritise relevant data transmission, the compression behaviour needs to be investigated. 

Synthetically creating images is currently based on parametric empirical models for directional reflectance properties of airless regolith surfaces, developed by Hapke (\cite{hapke1981bidirectional}, \cite{hapke1984bidirectional}, \cite{hapke1986bidirectional}, \cite{hapke2002bidirectional}, \cite{hapke2008bidirectional}, \cite{hapke2012bidirectional}). However, the Hapke model is also being challenged because of its poor performance in some areas \cite{shkuratov2012critical}. One of these areas are shadows. These are especially relevant to surface features of \gls{sssb}s. One possible solution is to use Ray-tracing or Path-tracing techniques.

In this thesis, the first implementation of \gls{sispo} is developed. The aim was not to complete development of the software package but to provide a first draft. The implementation covers the entire processing pipeline from initial 3D model, flyby simulation, rendering, compression and reconstruction to 3D model with some degree of automation. 
The software package should provide a possibility to simulate different trajectories. Rendering should provide as realistic output as possible. Furthermore, it is necessary to include rendering of various \gls{sssb}s, including different size, shape and surface features. \gls{sispo} is being developed to provide these functionalities in a single software package.


\newpage
Advancing the field requires improving the statistics on the different types of \gls{sssb}s \cite{Pajusalu2019CharacterizationMapping}. Several new mission ideas are aiming for this goal. However, technological challenges have to be overcome first. Additionally, new mission concepts like the \gls{ci} mission require a new approach to model \gls{sssb}s. \gls{ci} is a mission that will investigate either a dynamically new comet from the Öpik-Oort cloud or an interstellar object. The mission will consist of a main spacecraft and two sub-spacecrafts that will be launched to the \gls{l2} and wait until a suitable \gls{sssb} will pass by. Consequently, the target body will not be known before launch hence systems have to be developed to deal with a large variety of bodies. However, only few \gls{sssb}s have been visited by spacecraft and thus only very limited real images are available. A method to deal with such sparse available data is to synthetically create images of \gls{sssb}s.

Miniaturisation of technology enables completely new types of missions. With increasing capabilities of smaller spacecraft, they are becoming more and more considered for deep space missions. Either in mother-daughter spacecraft combinations (comet interceptor) or as swarms of satellites (MAT). Small satellites not only increase the number of spacecraft but also allow for probing more dangerous environments due to the decreased costs. However, this increase in deep space missions requires new methods for navigation and data transmission due to limited data budgets. Currently, all missions rely on a \gls{dsn} for both functions. The increasing number of spacecraft requires an increasing level of on-board autonomy in both, navigation and data compression for communication.

As a second part, \gls{sispo} can reconstruct 3D models from images via a \gls{sfm} approach. Since the data budget of small satellite missions is heavily limited, and there is a high risk of losing such a spacecraft due to the higher risk environments, maximising the science output becomes a problem of data compression and transmission. 
This thesis focuses on image compression and decompression. These effects will be quantified using the reconstructed 3D models against the "ideal" model. A reference instrument definition similar to a suggested instrument for the 'Comet Interceptor' mission. Different compression and decompression algorithms will be evaluated using different \gls{sssb}s. Furthermore, different spacecraft trajectories will be examined to investigate limitations of the simulation environment.
