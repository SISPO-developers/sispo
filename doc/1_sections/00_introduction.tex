\section{Introduction}
Studying \gls{sssb}s is getting more and more interesting. There are several aspects to this. From a scientific perspective, \gls{sssb}s are interesting, because they are unaltered remnants from the formation of the Solar System. Economically, \gls{sssb} are becoming interesting, as there are several materials that could be mined from them. Thirdly, planetary defence is interested in defending Earth from a possible threat by \gls{sssb}.

Currently, there is only a very limited knowledge on \gls{sssb} available. Most knowledge about there size, shape and colour is inferred from ground based observations and space telescopes. Another method is close-up observation through spacecraft. Only {\color{red}XXX} bodies have been visited by spacecraft, either with a fly-by {\color{red}(XXX)} or going into orbit around them {\color{red}(XXX)}. Considering the approx. ~{\color{red} XXXXXXX} known bodies shows how many more objects there are to be discovered. To gain a deeper understanding, several missions to \gls{sssb}s are planned. 

Miniaturisation of technology enables completely new types of missions. With increasing capabilities of smaller spacecraft, they are becoming more and more considered for deep space missions. Either in mother-daughter spacecraft combinations (comet interceptor) or as swarms of satellites (MAT). Small satellites not only increase the number of spacecraft but also allow for probing more dangerous environments due to the decreased costs. However, this increase in deep space missions requires a new methods for navigation and data transmission. Currently, all missions rely on a \gls{dsn} for both functions. The increasing number of spacecraft requires an increase of onboard autonomy in both, data compression and navigation.



Hello World \cite{Kohout2018FeasibilityStudy}
\gls{esa}