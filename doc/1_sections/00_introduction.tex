\section{Introduction}
Studying \gls{sssb}s is getting more and more interesting. There are several aspects to this. From a scientific perspective, \gls{sssb}s are interesting, because they are unaltered remnants from the formation of the Solar System. Economically, \gls{sssb} are becoming interesting, as there are several materials that could be mined from them. Thirdly, planetary defence is interested in defending Earth from a possible threat by \gls{sssb}.

Currently, there is only a very limited knowledge on \gls{sssb} available. Most knowledge about there size, shape and colour is inferred from ground based observations and space telescopes. Another method is close-up observation through spacecraft. Only {\color{red}XXX} bodies have been visited by spacecraft, either with a fly-by {\color{red}(XXX)} or going into orbit around them {\color{red}(XXX)}. Considering the approx. ~{\color{red} XXXXXXX} known bodies shows how many more objects there are to be discovered. Whilst it is possible to obtain a rough overall statistics of many \gls{sssb}s using ground and space telescopes, fly-by and missions going into orbit around an \gls{sssb} can provide a much more detailed analysis of these bodies through in-situ measurements and showing local variations of e.g. albedo.

Advancing the field requires improving the statistics on the different types of \gls{sssb}s \cite{Pajusalu2019CharacterizationMapping}. Several new mission ideas are aiming for this goal. However, technological challenges have to be overcome first. Additionally, new mission concepts like the \gls{ci} mission require a new approach to model \gls{sssb}s. \gls{ci} is a mission that will investigate either a dynamically new comet from the Öpik-Oort cloud or an interstellar object. The mission will consist of a main spacecraft and two sub-spacecrafts that will be launched to the \gls{l2} and wait until a suitable \gls{sssb} will pass by. Consequently, the target body will not be known before launch hence systems have to be developed to deal with a large variety of bodies. However, only few \gls{sssb}s have been visited by spacecraft and thus only very limited real images are available. A method to deal with such sparse available data is to synthetically create images of \gls{sssb}s.

Miniaturisation of technology enables completely new types of missions. With increasing capabilities of smaller spacecraft, they are becoming more and more considered for deep space missions. Either in mother-daughter spacecraft combinations (comet interceptor) or as swarms of satellites (MAT). Small satellites not only increase the number of spacecraft but also allow for probing more dangerous environments due to the decreased costs. However, this increase in deep space missions requires new methods for navigation and data transmission due to limited data budgets. Currently, all missions rely on a \gls{dsn} for both functions. The increasing number of spacecraft requires an increasing level of on-board autonomy in both, navigation and data compression for communication.

One step towards higher on-board autonomy is the prioritisation of images for transmission. With the  possibility of loss of spacecraft during proximity operations, the most relevant information needs to be transmitted first and later complemented with additional data if possible.

Furthermore, to increase navigation autonomy on-board deep space missions, optical navigation will be used for proximity operations. One problem in developing such system is the limited number and availability of real images. To solve this problem synthetic image generation can be used.

Within this work, the \gls{sispo} was developed. This simulation environment allows to create different 3D models of \gls{sssb}s using procedural surface generation. This approach allows to create different surface textures as well as different shapes for \gls{sssb}. These models can then be used to render photo-realistic images for a given \gls{sssb} orbit and spacecraft trajectory. Current methods mostly rely on an approximation to render e.g. the shadows on \gls{sssb}s. The proposed approach uses ray-tracing to generate photo-realistic images.

As a second part, \gls{sispo} can reconstruct 3D models from images via a \gls{sfm} approach. Since the data budget of small satellite missions is heavily limited, and there is a high risk of losing such a spacecraft due to the higher risk environments, maximising the science output becomes a problem of data compression and transmission. 
This thesis focuses on image compression and decompression. These effects will be quantified using the reconstructed 3D models against the "ideal" model. A reference instrument definition similar to a suggested instrument for the 'Comet Interceptor' mission. Different compression and decompression algorithms will be evaluated using different \gls{sssb}s. Furthermore, different spacecraft trajectories will be examined to investigate limitations of the simulation environment.